\chapter*{Aufgabenstellung zur Projektarbeit}\label{aufgabenstellung}
%\section {motivation}
Ziel dieser Arbeit ist es, einen Laufzeitmechanismus für hybride
Einbettungsverfahren in heterogenen, NoC-basierten Mehrprozessorsystemen
umzusetzen. Dazu müssen einerseits die grundlegenden Datenstrukturen und
Algorithmen bereitgestellt werden, siehe \cite{jaeger}. Diese sollen dann in der PGAS-Programmiersprache
X10 \cite{x10} umgesetzt werden. Letztendlich soll das umgesetzte
Verfahren durch Einsatz des \textit{InvadeSIM} -- Simulators \cite{cf:MPSoCs} simuliert werden,
und ausgewertet werden, wie es sich in einem realen Vielkernsystem verhalten
würden. Hierbei soll vor allem der mit dem Constraint-Solving verbundene Rechenaufwand
quantifiziert werden.
\\
\\
Im Rahmen der Arbeit sind folgende Arbeitsschritte zu tätigen:
\begin{itemize}
\item Einarbeitung in die Thematik und die grundlegenden Literatur.
\item Festlegung der Datenstrukturen und Algorithmen. Für die Arbeit ausreichend ist die Umsetzung der Min-Conflict-Heuristik (siehe \cite{jaeger}).
\item Deren Umsetzung in der Programmiersprache X10.
\item Simulation und experimentelle Auswertung des Laufzeitverhaltens mit Hilfe von InvadeSIM \cite{cf:MPSoCs}.
\item Erstellung einer schriftlichen Projektarbeit und der Dokumentation aller Programme. Archivierung der relevanten Daten auf einer CD/DVD.
\end{itemize}