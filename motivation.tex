\chapter{Motivation}\label{motivation}
%\section {motivation}
Zukünftige Mehrprozessorsysteme bestehen aus einer Vielzahl heterogener
Ressourcen [1], in denen die Kommunikation zwischen Ressourcen
durch Network-on-Chips (NoCs) ermöglicht wird (siehe z. B. [2]). Diese Technologie
unterstützt Anwendungsszenarien, in denen eine sehr große Zahl an Programmen
dynamisch starten und terminieren. Allerdings führt der Einsatz von
immer mehr gemeinsam genutzten, heterogenen Hardwareressourcen dazu, dass
die Vorhersagbarkeit nichtfunktionaler Eigenschaften der Ausführung eines Programms
erschwert wird. Dies betrifft z. B. den erwarteten Durchsatz, die Echtzeit
Fähigkeit, Zuverlässigkeits- oder Sicherheitseigenschaften.\\
\\
Neue Ansätze wie [3, 4] schlagen daher hybride Verfahren zur Einbettung von
Anwendungen (engl. hybrid application mapping) vor. Im Fokus stehen Programme
zur Bild- und Signalverarbeitung, die nach dem Starten Daten periodisch
verarbeiten. Zur Entwurfszeit wird im Rahmen einer Entwurfsraumexploration analysiert, welchen Einfluss verschiedene Allokationen heterogener
Ressourcen für die Programmausführung auf deren nichtfunktionale Eigenschaften
haben. Die Exploration evaluiert und optimiert dabei eine Vielzahl solcher
Implementierungsalternativen, wobei nur Alternativen beibehalten werden, die
bezüglich ihrer Zielgrößen nicht durch andere dominiert werden (sog. Betrieb-
spunkte [5]). Die Grundidee dieses Vorgehens ist, dass die ermittelten nichtfunktionalen
Eigenschaften einer Programmausführung garantiert werden können,
wenn die geforderten Ressourcen zur Laufzeit bereitgestellt werden. Hierbei
müssen nach [4] bestimmte Nebenbedingungen (Constraints) eingehalten werden:
Einerseits sind dies Bindungsanforderungen bezüglich der Ressourcentypen,
auf die die Anwendungstasks gebunden werden können. Andererseits bestehen
Routinganforderungen für die Realisierung der Datenabhängigkeiten zwischen
Anwendungstasks, z. B. bezüglich der maximalen Anzahl an Sprüngen (engl.
hops) durch das NoC oder der benötigten Bandbreite.