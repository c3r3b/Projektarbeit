\addchap*{Aufgabenstellung zur Bachelorarbeit}
Ziel dieser Arbeit ist es, einen Mechanismus für die \textit{invade}-Phase von Programmen mit strikten Ressourcenanforderungen zu entwerfen
und zu untersuchen. Ausgangspunkt ist ein heterogenes Mehrprozessorsystem,
dessen Prozessoren durch ein Network-on-Chip (NOC) \cite{mappingNocArchitectures} miteinander verbunden
sind. Außerdem sind periodisch ausgeführte Programme gegeben, die durch
Task-Graphen repräsentiert werden. Hierbei hat jeder Task Bindungsanforderungen bezüglich der Ressourcentypen, auf die er gebunden werden kann. Die
Datenabhängigkeiten zwischen Tasks haben Routinganforderungen, z. B. bezüglich
der maximalen Anzahl an Sprüngen (engl. \textit{hops}) durch das NoC oder der
Bandbreite. \\ \\
Das Problem der Selbsteinbettung lässt sich daher als Constraint-Satisfaction-Problem (CSP) \cite{foundationCSP} \cite{tutorialCSP} formulieren. Es gibt hier verschiedene Algorithmen, die
eingesetzt und teilweise kombiniert werden können, um systematisch (z. B. Backtracking-
Algorithmus, Forward-Checking-Algorithmus \cite{foundationCSP}) oder heuristisch (z. B.
Min-Conflicts-Algorithmus \cite{cspsolvingRepairMethod} nach einer Lösung zu suchen. Diese Algorithmen
beeinflussen also, ob das Verfahren überhaupt erkennt, wenn keine gültige
Lösung existiert, oder im anderen Fall, ob eine gültige Lösung gefunden wird
und wie lange der Suchlauf dann dauert. \\ \\
In dieser Arbeit sollen daher verschiedene Algorithmen vorgestellt, an das Problem
der Selbsteinbettung angepasst und dann bezüglich Erfolgsquote der Einbettung
und Einbettungszeit ausgewertet werden.